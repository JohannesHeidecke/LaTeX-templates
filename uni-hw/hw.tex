\title{Title}
\author{Author}
\date{Date}

\documentclass[12pt,a4paper]{article}

%% Language and font encodings
\usepackage[english]{babel}
\usepackage[utf8x]{inputenc}
\usepackage[T1]{fontenc}

%% Sets page size and margins
\usepackage[a4paper,top=3cm,bottom=2cm,left=3cm,right=3cm,marginparwidth=1.75cm]{geometry}

\usepackage[parfill]{parskip}
\usepackage[small]{titlesec}
\usepackage{amsmath}
\usepackage{graphicx}
\usepackage{subcaption}
\usepackage{natbib}
\usepackage{listings}  % for code snippets
\setlength{\parskip}{10pt plus 1pt minus 1pt}
\begin{document}
\maketitle

\begin{abstract}
Your abstract.
\end{abstract}

\section{Introduction}

Look at subfigure \ref{fig:sfig2} to see an example.

\begin{figure}[htb]
\begin{subfigure}{.5\textwidth}
  \centering
  \includegraphics[width=.8\linewidth]{img/dummy}
  \caption{1a}
  \label{fig:sfig1}
\end{subfigure}%
\begin{subfigure}{.5\textwidth}
  \centering
  \includegraphics[width=.8\linewidth]{img/dummy}
  \caption{1b}
  \label{fig:sfig2}
\end{subfigure}
\caption{Example figure for demonstration purposes}
\label{fig:fig}
\end{figure}


This is a citation: \cite{breiman2001random}.


\[ equation \cdot a \rightarrow \]

% https://en.wikibooks.org/wiki/LaTeX/Source_Code_Listings
\lstset{language=Pascal}
\begin{lstlisting}[frame=single]  % Start your code-block
for i:=maxint to 0 do
begin
{ do nothing }
end;
Write('Case insensitive ');
Write('Pascal keywords.');
\end{lstlisting}

\begin{figure}[htb]
\includegraphics[width=.8\linewidth]{img/dummy}	
\caption{1a}
\label{fig:fig2}
\end{figure}

\bibliographystyle{alpha}
\bibliography{ref/references.bib}

\end{document}